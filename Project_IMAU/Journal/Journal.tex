\documentclass[twocolumn,a4paper,aps,amsmath,amssymb,floatfix,superscriptaddress]{revtex4-2}
\usepackage{amsmath}
\usepackage{gensymb}
\usepackage{amsfonts}
\usepackage[a4paper, total={6in, 8in}]{geometry}
\usepackage{braket}
\usepackage{calligra}
\usepackage{cancel}
\usepackage{hhline}
\usepackage{tikz-feynman}
\usepackage{physics}
\usepackage{amsfonts}
\usepackage{makecell}
\usepackage{threeparttable}
\usepackage{multirow}
\usepackage{caption}
\usepackage{float}
\usepackage[caption = false]{subfig}
\usepackage{graphicx}     % Include figure files
\usepackage{dcolumn}		% Align table columns on decimal point
\usepackage{hyperref}     % To allow for hyperlinks in the document and to e.g. web sources
\usepackage{bm}
\usepackage[utf8]{inputenc}  % Allows use of één instead of \'e\'en.

\newcommand{\pref}{\frac{1}{4\pi\epsilon_0}}
\newcommand\at[3]{\left.#1\right|_{#2}^{#3}}
\renewcommand{\thesubsection}{\thesection.\roman{subsection}}
\renewcommand*{\thesubsubsection}{\thesubsection.\arabic{subsubsection}}
\renewcommand{\d}{\textnormal{d}}
\renewcommand{\r}{\textnormal{\large\textcalligra{r}}}
\newcommand{\del}{\mathbf{\nabla}}
\newcommand{\Dv}[3][1]{\frac{\textnormal{D}^{#1}#2}{\textnormal{D}#3^{#1}}}
\renewcommand{\vec}{\mathbf}

\begin{document}	
	\title{Lab Journal: analysis of an Atlantic clay sediment sample near the Canary Islands}
	\author{E. de Bie}
	\affiliation{Institute for Marine and Atmospheric Research Utrecht, \\
		Department of Physics, \\
		Faculty of Science, \\
		Utrecht University}
	
	\author{H. Hildebrand}
	\affiliation{Institute for Marine and Atmospheric Research Utrecht, \\
		Department of Physics, \\
		Faculty of Science, \\
		Utrecht University}
	
	\author{J. Gerlagh}
	\affiliation{Institute for Marine and Atmospheric Research Utrecht, \\
		Department of Physics, \\
		Faculty of Science, \\
		Utrecht University}
	\maketitle
	
	\section{Introduction}
	First, we shall discuss the points we are to research before starting our research plan. 
	In literature, we shall attempt to find:
	\begin{enumerate}
		\item the precipitation rate for this portion of the Atlantic, to determine the time resolution in our sample;
		\item perturbation of the ocean floor through natural or human mechanisms;
		\item concentration of microplastics, as an indication of the increase in nanoplastic concentration;
		\item amounts of detritus, and whether this would contaminate the measurement;
		\item protocols for extraction of the plastics from the clay.
	\end{enumerate}
	From this, we shall construct a protocol for our measurements and from there, we shall find the tools we may need to acquire.\\
	In general we split these into three categories: items 1 and 2 (to be researched by Joost) describe the context of the samples, items 3 and 4 (to be researched by Has) describe the contaminants, and item 5 (to be researched by Eva) describes the practical steps to be taken.
	
	\subsection{Question Answers}
	\subsubsection{}
	We found numerous different values for the sedimentation rates, influenced by exact location, period, etc. The found values were however all in the range of several mm \cite{Sediment distribution in the Atlantic} to cm \cite{Geomorphical investigaions NW Africa} per thousand years. So either way, assuming the pollution of the environment started, say, a century ago \cite{History of Plastic}, in theory only the upper millimeters of the ocean bottom should contain (traces of) plastic. Although that is unfortunately not the whole story.
	(There does perhaps seem to be more data, except we have not (yet?) figured out where exactly to find it or how to get it into a readable format, if we are even able to. Being this paper \cite{Atlantic sediment cores last 40k years}, but also the sites of several organisations, such as PLOCAN, BIOS and IODP)
	\subsubsection{}
	Something to take into consideration when analysing these samples, are possible perturbations caused by both human and natural causes. This could be for example anchors, strong water flows or (deep sea) animals mixing some of the upper layers. We found a source stating that the ocean floor in the area we're interested in is currently stable "Under static (gravitational) loading conditions" \cite{Sediment stability western Canary Islands}. However, human influences can also cause mixing and other perturbations in the sedimentation. \cite{Anthrophogenic influence sedimentation} \cite{Human impacts marine fossil record}. Unfortunately there seems to be no way to deal with this, other than maybe changing our hypothesis on the (maximum) depth we expect to find significant amounts of (nano)plastics.
	\subsubsection{}
	
	\subsubsection{}
	
	\subsubsection{Existing Protocols}
	There exists research on the filtration, dialysis, and ultrafiltration retention rates using polystyrene nanospheres between 1000 and 50 nm.\cite{Assessment of nanoplastic extraction}
	
	\begin{thebibliography}{10}
		
		\bibitem{Assessment of nanoplastic extraction}
		Albignac, M., Maria, E., De Oliveira, T., Roux, C., Goudouneche, D., Mingotaud, A. F., Bordeau, G., \& ter Halle, A. (2023). \textit{Assessment of nanoplastic extraction from natural samples for quantification purposes.} Environmental Nanotechnology, Monitoring \& Management, 20. https://doi.org/10.1016/j.enmm.2023.100862
		
		\bibitem{Anthrophogenic influence sedimentation}
		Cheng, Z., Wang, X. H., Jalón-Rojas, I., \& Liu, Y. (2019). Reconstruction of sedimentation changes under anthropogenic influence in a medium-scale estuary based on a decadal chronological framework. Estuarine Coastal and Shelf Science, 227, 106295. https://doi.org/10.1016/j.ecss.2019.106295
		
		\bibitem{Sediment distribution in the Atlantic} Ewing, M., Carpenter, G., Windisch, C., \& Ewing, J. (1973). Sediment distribution in the oceans: the Atlantic. Geological Society of America Bulletin, 84(1), 71. https://doi.org/10.1130/0016-7606(1973)84
		
		\bibitem{Human impacts marine fossil record}
		Nawrot, R., Zuschin, M., Tomašových, A., Kowalewski, M., \& Scarponi, D. (2024). Ideas and perspectives: Human impacts alter the marine fossil record. Biogeosciences, 21(9), 2177–2188. https://doi.org/10.5194/bg-21-2177-2024
		
		\bibitem{Sediment stability western Canary Islands} Roberts, J. A., \& Cramp, A. (1996). Sediment stability on the western flanks of the Canary Islands. Marine Geology, 134(1–2), 13–30. https://doi.org/10.1016/0025-3227(96)00021-7
		
		\bibitem{History of Plastic} Verney, F. (2023, November 13). The history of plastic in 15 key dates. Carbiolice. https://www.carbiolice.com/en/news/the-history-of-plastic-in-15-key-dates-2/
		
		\bibitem{Geomorphical investigaions NW Africa} Von Suchodoletz, H., Faust, D., \& Zöller, L. (2008). Geomorphological investigations of sediment traps on Lanzarote (Canary Islands) as a key for the interpretation of a palaeoclimate archive off NW Africa. Quaternary International, 196(1–2), 44–56. https://doi.org/10.1016/j.quaint.2008.03.014
		
		\bibitem{Atlantic sediment cores last 40k years}
		Waelbroeck, C., Lougheed, B.C., Vazquez Riveiros, N. et al. Consistently dated Atlantic sediment cores over the last 40 thousand years. Sci Data 6, 165 (2019). https://doi.org/10.1038/s41597-019-0173-8
		
	\end{thebibliography}
	
\end{document}
