\documentclass[twocolumn,a4paper,aps,amsmath,amssymb,floatfix,superscriptaddress]{revtex4-2}
\usepackage{amsmath}
\usepackage{gensymb}
\usepackage{amsfonts}
\usepackage[a4paper, total={6in, 8in}]{geometry}
\usepackage{braket}
\usepackage{calligra}
\usepackage{cancel}
\usepackage{hhline}
\usepackage{tikz-feynman}
\usepackage{physics}
\usepackage{amsfonts}
\usepackage{makecell}
\usepackage{threeparttable}
\usepackage{multirow}
\usepackage{caption}
\usepackage{float}
\usepackage[caption = false]{subfig}
\usepackage{graphicx}     % Include figure files
\usepackage{dcolumn}		% Align table columns on decimal point
\usepackage{hyperref}     % To allow for hyperlinks in the document and to e.g. web sources
\usepackage{bm}
\usepackage[utf8]{inputenc}  % Allows use of één instead of \'e\'en.

\newcommand{\pref}{\frac{1}{4\pi\epsilon_0}}
\newcommand\at[3]{\left.#1\right|_{#2}^{#3}}
\renewcommand{\thesubsection}{\thesection.\roman{subsection}}
\renewcommand*{\thesubsubsection}{\thesubsection.\arabic{subsubsection}}
\renewcommand{\d}{\textnormal{d}}
\renewcommand{\r}{\textnormal{\large\textcalligra{r}}}
\newcommand{\del}{\mathbf{\nabla}}
\newcommand{\Dv}[3][1]{\frac{\textnormal{D}^{#1}#2}{\textnormal{D}#3^{#1}}}
\renewcommand{\vec}{\mathbf}

\begin{document}	
	\title{Lab Journal: analysis of an Atlantic clay sediment sample near the Canary Islands}
	\author{E. de Bie}
	\affiliation{Institute for Marine and Atmospheric Research Utrecht, \\
		Department of Physics, \\
		Faculty of Science, \\
		Utrecht University}
	
	\author{H. Hildebrand}
	\affiliation{Institute for Marine and Atmospheric Research Utrecht, \\
		Department of Physics, \\
		Faculty of Science, \\
		Utrecht University}
	
	\author{J. Gerlagh}
	\affiliation{Institute for Marine and Atmospheric Research Utrecht, \\
		Department of Physics, \\
		Faculty of Science, \\
		Utrecht University}
	\maketitle
	
	\section{Introduction}
	First, we shall discuss the points we are to research before starting our research plan. 
	In literature, we shall attempt to find:
	\begin{enumerate}
		\item the precipitation rate for this portion of the Atlantic, to determine the time resolution in our sample;
		\item perturbation of the ocean floor through natural or human mechanisms;
		\item concentration of microplastics, as an indication of the increase in nanoplastic concentration;
		\item amounts of detritus, and whether this would contaminate the measurement;
		\item protocols for extraction of the plastics from the clay.
	\end{enumerate}
	From this, we shall construct a protocol for our measurements and from there, we shall find the tools we may need to acquire.\\
	In general we split these into three categories: items 1 and 2 (to be researched by Joost) describe the context of the samples, items 3 and 4 (to be researched by Has) describe the contaminants, and item 5 (to be researched by Eva) describes the practical steps to be taken.
	
	\subsection{Question Answers}
	\subsubsection{}
	We found numerous different values for the sedimentation rates, influenced by exact location, period, etc. The found values were however all in the range of several mm  to cm  per thousand years. So either way, assuming the pollution of the environment started, say, a century ago , in theory only the upper millimeters of the ocean bottom should contain (traces of) plastic. Although that is unfortunately not the whole story.
	(There does perhaps seem to be more data, except we have not (yet?) figured out where exactly to find it or how to get it into a readable format, if we are even able to. Being this paper , but also the sites of several organisations, such as PLOCAN, BIOS and IODP)
	\subsubsection{}
	Something to take into consideration when analysing these samples, are possible perturbations caused by both human and natural causes. This could be for example anchors, strong water flows or (deep sea) animals mixing some of the upper layers. We found a source stating that the ocean floor in the area we're interested in is currently stable "Under static (gravitational) loading conditions" . However, human influences can also cause mixing and other perturbations in the sedimentation.  . Unfortunately there seems to be no way to deal with this, other than maybe changing our hypothesis on the (maximum) depth we expect to find significant amounts of (nano)plastics.
	
	\subsubsection{Microplastic concentration as proxy for nanoplastic concentration}
	It was found that microplastic concentration is not a valid approximation for nanoplastic concentration. Microplastics arrive at the sea floor with different mechanisms and therefore do not have an exact relation to nanoplastics.   Microplastics behave like particles and are capable of sinking. Nanoplastics behave like colloids and do not sink quickly, unless they (self)aggregate.  The aggregates are too big to pass through the less porous filters. This can lead to significant loss of nanoplastics in the sample. The aggregates can be destroyed using sonication.  
	\subsubsection{Significance of detritus in the sample}
	Detritus is present on the seafloor, as it is everywhere in the ocean. Detritus is also one of the things that the nanoplastics can aggregate on. As the aggregates are dispersed using sonication and afterwards the supernatant is filtered only very low mass organic compounds should remain in solution. These typically have a low boiling point so should not be detected at the same time as the nanoplastics (high boiling point). {p. 79} If they do somehow come along with the nanoplastics (if the nanoplastics aggregate to them again during the rise in temperature perhaps) then their MS footprints should be sufficiently different as to not warrant any concern.
	\subsubsection{Existing Protocols}
	There exists research on the filtration, dialysis, and ultrafiltration retention rates using polystyrene nanospheres between 1000 and 50 nm.\\
	From this research, we get several takeaways:\\
	Polymer membranes are not recommended, with retention rates far too high.\\
	"there was always retention even if the porosity was up to 50 times larger than the beads diameter. This implies that for pre-filtration, in the view to NP quantification in a real sample, the recoveries would be insufficient and very difficult to rationalize because of the high particle heterogeneity."\\
	Dialysis led to important losses of 50\% after 48h and 70\% after 72h.\\
	Ultrafiltration had losses of 35-40\%\\
	\\
	Next, I found a publication which determined metal fractionation of a marine sediment core from Antarctica.\\
	Again, there were several takeaways:\\
	Stainless steel was used to sample the core. It was subsampled using a Teflon blade. Only the inner part was analysed.\\
	To determine moisture content, a subsample was dried at 105° C until the mass was constant.\\
	The storage containers were cleaned extensively. They used a clean room (Class-100.000 / ISO 8, the "dirtiest" clean room).\\
	\\
	Next, I analysed section 2 (`Materials and method') of a publication in which a cire sample was analysed from the Indian Ocean.\cite{IndianOceanCore}
	In this, they used acetic acid to remove inorganic carbon contents of the sediment, and hydrogen peroxide to remove organic carbon contents. After this, they were washed with deionized water to remove remaining aces, peroxide, and salt.\\
	\\
	Finally, I read a paper of the PASADO core processing strategy, which appeared to have been made to allow several disciplines to perform research on the same core. Still, I extracted some points that I thought might be useful.\\
	Namely, the idea of leaving one half of the core (if split lengthways) for archiving purposes, if we have enough of the core to do that. Secondly, the idea to use a cutting tool with the right shape to not take the sides of the core, as those might be contaminated.\\
	\\
	In summary:\\
	We should try to avoid any filtering with porosity less than two orders of magnitude above 1 $\mu$m. If needed, stainless steel grids with cut off at 5 or 10 $\mu$m are preferred, with 15-20\% retention.\\
	We should use steel and glass tools wherever possible, as to not contaminate the sample.
	We clean the bottles using temperature. Verify if we can get a glovebox.\\
	Peroxide can be used to remove organic carbon contents, acetic acid can be used to remove inorganic carbon contents.\\
	A D-shaped cutter is useful as not to incorporate plastics from the storage medium. Furthermore, to use only one half of the sample may be a good idea for insurance's sake.
	
	\bibliography{JournalNotes}{}
	\bibliographystyle{plain}
\end{document}
