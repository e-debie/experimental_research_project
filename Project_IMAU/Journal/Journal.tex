\documentclass[twocolumn,a4paper,aps,amsmath,amssymb,floatfix,superscriptaddress]{revtex4-2}
\usepackage{amsmath}
\usepackage{gensymb}
\usepackage{amsfonts}
\usepackage[a4paper, total={6in, 8in}]{geometry}
\usepackage{braket}
\usepackage{calligra}
\usepackage{cancel}
\usepackage{hhline}
\usepackage{tikz-feynman}
\usepackage{physics}
\usepackage{amsfonts}
\usepackage{makecell}
\usepackage{threeparttable}
\usepackage{multirow}
\usepackage{caption}
\usepackage{float}
\usepackage[caption = false]{subfig}
\usepackage{graphicx}     % Include figure files
\usepackage{dcolumn}		% Align table columns on decimal point
\usepackage{hyperref}     % To allow for hyperlinks in the document and to e.g. web sources
\usepackage{bm}
\usepackage[utf8]{inputenc}  % Allows use of één instead of \'e\'en.

\newcommand{\pref}{\frac{1}{4\pi\epsilon_0}}
\newcommand\at[3]{\left.#1\right|_{#2}^{#3}}
\renewcommand{\thesubsection}{\thesection.\roman{subsection}}
\renewcommand*{\thesubsubsection}{\thesubsection.\arabic{subsubsection}}
\renewcommand{\d}{\textnormal{d}}
\renewcommand{\r}{\textnormal{\large\textcalligra{r}}}
\newcommand{\del}{\mathbf{\nabla}}
\newcommand{\Dv}[3][1]{\frac{\textnormal{D}^{#1}#2}{\textnormal{D}#3^{#1}}}
\renewcommand{\vec}{\mathbf}


\begin{document}	
	\title{Lab Journal: analysis of an Atlantic clay sediment sample near the Canary Islands}
	\author{E. de Bie}
	\affiliation{Institute for Marine and Atmospheric Research Utrecht, \\
		Department of Physics, \\
		Faculty of Science, \\
		Utrecht University}
	
	\author{H. Hildebrand}
	\affiliation{Institute for Marine and Atmospheric Research Utrecht, \\
		Department of Physics, \\
		Faculty of Science, \\
		Utrecht University}
	
	\author{J. Gerlagh}
	\affiliation{Institute for Marine and Atmospheric Research Utrecht, \\
		Department of Physics, \\
		Faculty of Science, \\
		Utrecht University}
	\maketitle
	
	\section{Introduction}
	First, we shall discuss the points we are to research before starting our research plan. 
	In literature, we shall attempt to find:
	\begin{enumerate}
		\item the sedimentation rate for this portion of the Atlantic, to determine the time resolution in our sample;
		\item perturbation of the ocean floor through natural or human mechanisms;
		\item concentration of microplastics, as an indication of the increase in nanoplastic concentration;
		\item amounts of detritus, and whether this would contaminate the measurement;
		\item protocols for extraction of the plastics from the clay.
	\end{enumerate}
	From this, we shall construct a protocol for our measurements and from there, we shall find the tools we may need to acquire.\\
	In general we split these into three categories: items 1 and 2 (to be researched by Joost) describe the context of the samples, items 3 and 4 (to be researched by Has) describe the contaminants, and item 5 (to be researched by Eva) describes the practical steps to be taken.
	
	\subsection{Question Answers}
	\subsubsection{Sedimentation rate}
	We found numerous different values for the sedimentation rates, influenced by exact location, period, etc. The found values were however all in the range of several mm \cite{Sediment distribution in the Atlantic} to cm \cite{Geomorphical investigaions NW Africa} per thousand years. So either way, assuming the pollution of the environment started, say, a century ago \cite{History of Plastic}, in theory only the upper millimeters of the ocean bottom should contain (traces of) plastic. Although that is unfortunately not the whole story.
	(There does perhaps seem to be more data, except we have not (yet?) figured out where exactly to find it or how to get it into a readable format, if we are even able to. Being this paper \cite{Atlantic sediment cores last 40k years}, but also the sites of several organisations, such as PLOCAN, BIOS and IODP)
	\subsubsection{Perturbation of the ocean floor}
	Something to take into consideration when analysing these samples, are possible perturbations caused by both human and natural causes. This could be for example anchors, strong water flows or (deep sea) animals mixing some of the upper layers. We found a source stating that the ocean floor in the area we're interested in is currently stable "Under static (gravitational) loading conditions" \cite{Sediment stability western Canary Islands}. However, human influences can also cause mixing and other perturbations in the sedimentation. \cite{Anthrophogenic influence sedimentation} \cite{Human impacts marine fossil record}. Unfortunately there seems to be no way to deal with this, other than maybe changing our hypothesis on the (maximum) depth we expect to find significant amounts of (nano)plastics.
	
	\subsubsection{Microplastic concentration as proxy for nanoplastic concentration}
	It was found that microplastic concentration is not a valid approximation for nanoplastic concentration. Microplastics arrive at the sea floor with different mechanisms and therefore do not have an exact relation to nanoplastics. \cite{Vertical flux of microplastic} \cite{MP in Atlantic deep waters} Microplastics behave like particles and are capable of sinking. Nanoplastics behave like colloids and do not sink quickly, unless they (self)aggregate. \cite{Aggregation of NP what we know} The aggregates are too big to pass through the less porous filters. This can lead to significant loss of nanoplastics in the sample. The aggregates can be destroyed using sonication. \cite{Heteroaggregation disaggregation} 
	\subsubsection{Significance of detritus in the sample}
	Detritus is present on the seafloor, as it is everywhere in the ocean. Detritus is also one of the things that the nanoplastics can aggregate on. As the aggregates are dispersed using sonication and afterwards the supernatant is filtered only very low mass organic compounds should remain in solution. These typically have a low boiling point so should not be detected at the same time as the nanoplastics (high boiling point). \cite{McMurry}{p. 79} If they do somehow come along with the nanoplastics (if the nanoplastics aggregate to them again during the rise in temperature perhaps) then their MS footprints should be sufficiently different as to not warrant any concern.
	\subsubsection{Existing Protocols}
	There exists research on the filtration, dialysis, and ultrafiltration retention rates using polystyrene nanospheres between 1000 and 50 nm.\cite{Assessment of nanoplastic extraction}\\
	From this research, we get several takeaways:\\
	Polymer membranes are not recommended, with retention rates far too high.\\
	"there was always retention even if the porosity was up to 50 times larger than the beads diameter. This implies that for pre-filtration, in the view to NP quantification in a real sample, the recoveries would be insufficient and very difficult to rationalize because of the high particle heterogeneity."\\
	Dialysis led to important losses of 50\% after 48h and 70\% after 72h.\\
	Ultrafiltration had losses of 35-40\%\\
	\\
	Next, I found a publication which determined metal fractionation of a marine sediment core from Antarctica.\cite{Geochemical characterization Antarctica}\\
	Again, there were several takeaways:\\
	Stainless steel was used to sample the core. It was subsampled using a Teflon blade. Only the inner part was analysed.\\
	To determine moisture content, a subsample was dried at 105° C until the mass was constant.\\
	The storage containers were cleaned extensively. They used a clean room (Class-100.000 / ISO 8, the "dirtiest" clean room\cite{cleanrooms}).\\
	\\
	Next, I analysed section 2 (`Materials and method') of a publication in which a cire sample was analysed from the Indian Ocea.\cite{Core Indian Ocean}
	In this, they used acetic acid to remove inorganic carbon contents of the sediment, and hydrogen peroxide to remove organic carbon contents. After this, they were washed with deionized water to remove remaining aces, peroxide, and salt.\\
	\\
	Finally, I read a paper of the PASADO core processing strategy, which appeared to have been made to allow several disciplines to perform research on the same core. Still, I extracted some points that I thought might be useful.\\
	Namely, the idea of leaving one half of the core (if split lengthways) for archiving purposes, if we have enough of the core to do that. Secondly, the idea to use a cutting tool with the right shape to not take the sides of the core, as those might be contaminated.\\
	\\
	In summary:\\
	We should try to avoid any filtering with porosity less than two orders of magnitude above 1 $\mu$m. If needed, stainless steel grids with cut off at 5 or 10 $\mu$m are preferred, with 15-20\% retention.\\
	We should use steel and glass tools wherever possible, as to not contaminate the sample.
	We clean the bottles using temperature. Verify if we can get a glovebox.\\
	Peroxide can be used to remove organic carbon contents, acetic acid can be used to remove inorganic carbon contents.\\
	A D-shaped cutter is useful as not to incorporate plastics from the storage medium. Furthermore, to use only one half of the sample may be a good idea for insurance's sake.
	
	\section{Practical}
	\subsection{Day 1 (May 22nd 2025)}
	\paragraph*{Research preparation}
	Before starting the practical, 4 beakers, which had been labeled from A to D (for 4 subsamples) and a box cutter with some extra blades (to cut the sample into subsamples) were heated in an oven for a sufficient time. The soil sample had in turn been frozen. The sonicator was cleaned by wiping it with moist (just regular tap water) paper towels and then rinsing it with distilled water. Also all the beakers were weighed. See Table \ref{tab:Sample_masses} for the found masses.
	\paragraph*{Blanks}
	To start, some blanks were prepared. To do so, first about 15-20 mL of HPLC water was brought into beaker B, 10-15 mL of which was immediately brought into a syringe, which was in turn put in the freezer. This was the syringe blank, SB for short. Then, beaker B was rinsed with a little bit of HPLC water, which was then transferred to a vial (when we say `vial', we refer to an approximately 5 mL vial compatible with our PTR-MS autosampler). This was the beaker blank, BB for short. Next, some HPLC water was poured directly from the bottle in another vial. This was the HPLC blank, HB for short. Lastly, an ``empty'' vial (meaning filled with air) was labelled our system blank, SB for short.
	\paragraph*{Subsample preparation}
	First, the sample was taken out of the freezer, in order for it to melt a bit. There was about 9 cm of soil in the syringe. As the outside of the syringe was dirty, it was scraped using a box cutter. In order to prevent contamination, the bottom of the sample was scraped off. \\
	The sample was then slowly pushed out the syringe, and cut with the aforementioned heated box cutters into 4 subsamples. They were, starting at the bottom, 3, 2, 2 and 2 cm thick and were transferred into beakers B, C, D and A respectively (also see Table \ref{tab:Sample_masses} and Figure \ref{fig:core-schematic}). The top layer of the soil was left in the syringe, in order to prevent any contamination. For each subsample, a new box cutter blade was used. The plan was to also scrape of the outer layer of each sample, as the contact with the syringe might have caused contamination and mixing. This was forgotten at first, but as nothing had been done to the subsamples when this mistake was realised, it was still done. All the subsamples were then weighed in their respective beakers, in order to determine the weight of all the subsamples (see Table \ref{tab:Sample_masses}).\\
	\begin{figure}[t!]
		\centering
		\includegraphics[width=0.7\linewidth]{"Images/Core schematic"}
		\caption{Schematic view of the syringe including core sample. As you go down in the image, you go deeper into the ocean floor. Note that the beaker A subsample does not reach the top of the syringe.}
		\label{fig:core-schematic}
	\end{figure}
	
	\begin{table*}
		\centering
		\begin{threeparttable}
			\caption[table]{Table1, bippity boppity}
			\label{tab:Sample_masses}
			\begin{tabular}{c||c|c|c|c}
				\textbf{Beaker} & A & B & C & D \\
				\hhline{=====}
				\textbf{Weight beaker (g)} & $69.2501$ & $65.0693$ & $68.4206$ & $61.1864$ \\
				\hline
				\textbf{Weight beaker + sample (g)} & $83.3133$ & $87.4262$ & $84.9395$ & $75.4117$ \\
				\hline
				\textbf{Weight sample (g)} & $14.0632$ & $22.3569$ & $16.5189$ & $14.2263$ \\
			\end{tabular}
		\end{threeparttable}
	\end{table*}
	
	In order to dissolve all the subsamples, 25 mL of HPLC water was added to all the beakers, using a graduated cylinder and a pipette. As beaker C was a 150 mL beaker, contrary to the other beakers, some of which were 100 mL beakers, that subsample was not fully submerged in the HPLC water. So, 25 mL more was added to all the beakers, bringing the total to 50 mL HPLC water per subsample. \\ The cylinderw as also rinsed with a little bit of HPLC water, which was then transferred to a vial, as a cylinder blank (CB).
	To prepare the beakers for the upcoming sonication, they were covered with aluminum foil, which was kept in place with some binding wire. They were then attached to the sonicator, which was then turned on for about 6 hours.
	
	\subsection{Day 2 (May 27th 2025)}
	\paragraph*{Between days}
	So the samples had been sonicated for about 6 hours May 22nd and again about 9 to 10 hours on May 26th. Afterwards, they had been left in a warm waterbath overnight. Additionally, three volumetric flasks (two 25 mL, one 50 mL) were washed with water and ethanol. 
	\paragraph*{Subsample filtration}
	Beaker A was removed from the sonicator. It contained a little over 60 mL of a very murky, clay-brown liquid. 
	Then about 70 mL of HPLC water was poured into a 150 mL beaker. 5 mL of this HPLC water was sucked up with a syringe, of which around 10 droplets (corresponding to approximately 200 \mu L) were transferred into a vial. This was the flask syringe blank, FSRB for short. The rest of the water inside the syringe was thrown away. Something similar was done to make another blank, this time however through a filter and excluding the first few droplets. This vial was labeled the filter blank, FLB for short.
	With a new syringe, approximately 10 mL of subsample A was transferred through a filter into a (10 mL) vial. The filter was replaced twice because it got too dirty, after roughly 2 mL each time. After the third filter was used up, the remaining subsample in the syringe was thrown away, and the process was repeated until the vial was full. This vial was marked vial A. 
	This process of filtering the subsamples and putting the filtrates in vials was repeated for subsamples B, C and D as well. Each time, a clean syringe was used. Contrary to A, the other subsamples all contained a two-layer system, rather than a (seemingly) homogenous liquid. All of them had a bottom layer of clay and a top layer of a slightly murky, light-brown liquid. These layers did however vary in size. For subsample B, the solid layer was a little over 20 mL in volume, and the liquid layer approximately 50 mL. For subsample C these layers were both roughly 25 mL in size, and for subsample D a little less than 20 mL solid and a little less than 30 mL liquid. For the subsamples B, C and D, just one filter was needed before their respective vial was full.
	All the vials (A-D and the blanks) were put in the freezer.
	\paragraph*{Subsubsample preparation}
	Next, the subsubsample preparation, as shown in Table \ref{tab:Subsample_make-up}, was started. First, vials with just the three plastics on the right were prepared. This was done in triplo and for every subsample, so in total $3 \cdot 4 \cdot 4 = 48$ vials were prepared (as samples $C_0$ do not require any additional plastics). All the vials were labeled as $C_n^i$, where $i = a,b,c$ is used as the the subsubsamples were prepared in triplo. 
	To prepare these subsubsamples, three vials filled with the necessary plastics (i.e. PET, PE and PP, see Table \ref{tab:Subsample_make-up}) dissolved in strong acids were used as the source. To do so, first three vials were labeled with $C_1^a$, $C_1^b$ and $C_1^c$. They were then all filled with the correct amount of PET, then PP and finally PE. After that was done, a cap was put on the vials and they were put away. The same was then done for subsubsamples $C_2$, $C_3$ and $C_4$. This will be defined as one "set". Throughout each set, the same pipette tips were used for each plastic and then in between sets these tips were replaced. Four sets were prepared using this procedure: one for each subsample. During the preparation, once the pipette point of the PE fell into one of the subsubsamples $C_2^a$. This vial was not marked, as it is assumed this will not have a significant impact on the results. 
	After the preparation was done, all the 48 vials were placed in a dessicator. There, it was found that in one set, two vials were labeled $C_3^c$ and none $C_2^c$. These two vials were marked.
	\paragraph*{Subsample storage}
	After slightly more than 3 hours all samples were removed from the dessicator and checked for the amount of liquid still present. Those that had very little liquid ($\~1/5$ of the vials) were set aside. All other vials were put back in the dessicator for 30 minutes. The vials set aside had their cap changed from one with 2 holes to a mostly airtight PTFE one. This is to prevent contamination from airborne NP. After the 30 minute dessication all other vials underwent the same procedure. The vials were set aside in the fume booth until the next lab day. 
	
	\begin{table*}
		\centering
		\begin{threeparttable}
			\caption[table]{Table2, bïppïtÿ, böppïtÿ. All numbers are in \mu L}
			\label{tab:Subsample_make-up}
			\begin{tabular}{c||c|c|c|c|c|c}
				\textbf{Subsample} & \textbf{Ocean sample } & \textbf{PS (1-1)} & \textbf{PVC (1-1)} & \textbf{PET (1-2)} & \textbf{PP (1-2)} & \textbf{PE (1-2)} \\
				\hhline{=======}
				 $\mathbf{C_0}$ & $500$ & $0$ & $0$ & $0$& $0$ & $0$ \\
				\hline
				  $\mathbf{C_1}$ & $500$ & $10$ & $30$ & $10$ & $20$ & $10$ \\
				\hline
				 $\mathbf{C_2}$ & $500$ & $20$ & $10$ & $5$ & $5$ & $20$ \\
				\hline
				 $\mathbf{C_3}$ & $500$ & $30$ & $40$ & $20$ & $15$ & $15$ \\
				\hline
				 $\mathbf{C_4}$ & $500$ & $40$ & $20$ & $15$ & $10$ & $5$ \\			
			\end{tabular}
		\end{threeparttable}
	\end{table*}
	
	\subsection{Day 3 (4th of June 2025)}
	When all the subsamples had been dessicated sufficiently, the 3 remaining components were added, namely the actual samples and the plastics PS and PVC (see Table \ref{tab:Subsample_make-up}, the three columns on the left labeled 1-1). This was done in similar fashion to the addition of the first three plastics, in the order D, B, C, A. This time however, all the subsamples also gained an extra letter A-D in front, as now there was a difference between for example $AC_1^a and BC_1^a$. Additionally, also the new subsamples $C_0$ (see Table \ref{tab:Subsample_make-up}) were prepared (also in triplo) of each sample. Again, the subsamples were prepared per set and then per subsample type. The ocean samples were first added to each subsample, then the correct amount of PS and PVC respectively. 
	After each set, all the pipette points were replaced. The pipette point fell in both subsample $CC_2^a$ and $AC_1^b$, after which it was replaced both times. As mentioned before, there was one $C_3^c$ too many and one $C_2^c$ too little. These were both marked with a star and put in the same set, namely A. The one which was treated as a $C_2$ was labeled with an extra 2 in brackets behind it (so they were labeled $AC_3^{c*}$ and $AC_3^{c*} (2)$). When all the subsamples were done, they were again placed in the dessicator starting at 11:35.
	After 4 hours of dessicating the subsamples were taken from the dessicator and the caps exchanged from perforated for the dessicator to PTFE for storage. Left to stand outside the flow booth (though covered) until next morning.
	
	\subsection{Day 4 (5th of June 2025)}
	Today the PTFE caps used for storage where changed to caps sealed by "sterile" aluminum foil and placed in the que for the PTR-MS machine. Caps for sample sets D and C were done by Nemat. A and B by Has. 
	
	\begin{thebibliography}{99}		
		\bibitem{Assessment of nanoplastic extraction}
		Albignac, M., Maria, E., De Oliveira, T., Roux, C., Goudouneche, D., Mingotaud, A. F., Bordeau, G., \& ter Halle, A. (2023). Assessment of nanoplastic extraction from natural samples for quantification purposes. \textit{Environmental Nanotechnology, Monitoring \& Management, 20}. https://doi.org/10.1016/j.enmm.2023.100862
		
		\bibitem{Geochemical characterization Antarctica}
		Burgay, F., Abollino, O., Buoso, S., Costa, E., Giacomino, A., La Gioia, C., Garofalo, S. F., Pecoraro, G., \& Malandrino, M. (2020). Geochemical characterization of a marine sediment core from the Joides Basin, Ross Sea, Antarctica. \textit{Marine Geology, 428}. https://doi.org/10.1016/j.margeo.2020.106286 
				
		\bibitem{cleanrooms}
		American Cleanroom Systems. (2024, September 4). \textit{Clean room Classifications \& ISO Standards | American Cleanrooms Systems.} https://www.americancleanrooms.com/cleanroom-classifications/
		
		\bibitem{Heteroaggregation disaggregation}
		Li, L., Luo, D., Luo, S., Yue, J., Li, X., Chen, L., Chen, X., Wen, B., Luo, X., Li, Y., Huang, W., \& Chen, C. (2024). Heteroaggregation, disaggregation, and migration of nanoplastics with nanosized activated carbon in aquatic environments: Effects of particle property, water chemistry, and hydrodynamic condition. \textit{Water Research, 266}. https://doi.org/10.1016/j.watres.2024.122399
		
		\bibitem{Vertical flux of microplastic}
		Rowlands, E., Galloway, T., Cole, M., Peck, V. L., Posacka, A., Thorpe, S., \& Manno, C. (2023). Vertical flux of microplastic, a case study in the Southern Ocean, South Georgia. \textit{Marine Pollution Bulletin, 193}. https://doi.org/10.1016/j.marpolbul.2023.115117
		
		\bibitem{MP in Atlantic deep waters}
		Mateos-Cárdenas, A., Wheeler, A. J., \& Lim, A. (2024). Microplastics and cellulosic microparticles in North Atlantic deep waters and in the cold-water coral Lophelia pertusa. \textit{Marine Pollution Bulletin, 206}. https://doi.org/10.1016/j.marpolbul.2024.116741
		
		\bibitem{Aggregation of NP what we know}
		Pradel, A., Catrouillet, C., \& Gigault, J. (2023). The environmental fate of nanoplastics: What we know and what we need to know about aggregation. \textit{NanoImpact, 29}. https://doi.org/10.1016/j.impact.2023.100453
		
		\bibitem{McMurry}
		McMurry, J. E. (2016). \textit{Organic Chemistry} (9th ed.). Cengage Learning.
		
		\bibitem{Core Indian Ocean}
		Sensarma, S., Gupta, S.M., Banerjee, R. et al. Change of lithofacies in marine sediment core from Quaternary to Pre-Quaternary: A case study from the Central Indian Ocean Basin. \textit{J Earth Syst Sci 129, 54} (2020). https://doi.org/10.1007/s12040-019-1318-z 
		
		\bibitem{PASADO}
		Ohlendorf, C., Gebhardt, C., Hahn, A., Kliem, P., \& Zolitschka, B. (2011). The PASADO core processing strategy — A proposed new protocol for sediment core treatment in multidisciplinary lake drilling projects. \textit{Sedimentary Geology, 239(1-2), 104–115}. https://doi.org/10.1016/j.sedgeo.2011.06.007
		
		\bibitem{Anthrophogenic influence sedimentation}
		Cheng, Z., Wang, X. H., Jalón-Rojas, I., \& Liu, Y. (2019). Reconstruction of sedimentation changes under anthropogenic influence in a medium-scale estuary based on a decadal chronological framework. \textit{Estuarine Coastal and Shelf Science, 227, 106295}. https://doi.org/10.1016/j.ecss.2019.106295
		
		\bibitem{Sediment distribution in the Atlantic} Ewing, M., Carpenter, G., Windisch, C., \& Ewing, J. (1973). Sediment distribution in the oceans: the Atlantic. \textit{Geological Society of America Bulletin, 84(1), 71}. https://doi.org/10.1130/0016-7606(1973)84
		
		\bibitem{Human impacts marine fossil record}
		Nawrot, R., Zuschin, M., Tomašových, A., Kowalewski, M., \& Scarponi, D. (2024). Ideas and perspectives: Human impacts alter the marine fossil record. \textit{Biogeosciences, 21(9), 2177–2188}. https://doi.org/10.5194/bg-21-2177-2024
		
		\bibitem{Sediment stability western Canary Islands} Roberts, J. A., \& Cramp, A. (1996). Sediment stability on the western flanks of the Canary Islands. \textit{Marine Geology, 134(1–2), 13–30}. https://doi.org/10.1016/0025-3227(96)00021-7
		
		\bibitem{History of Plastic} Verney, F. (2023, November 13). The history of plastic in 15 key dates. \textit{Carbiolice.} https://www.carbiolice.com/en/news/the-history-of-plastic-in-15-key-dates-2/
		
		\bibitem{Geomorphical investigaions NW Africa} Von Suchodoletz, H., Faust, D., \& Zöller, L. (2008). Geomorphological investigations of sediment traps on Lanzarote (Canary Islands) as a key for the interpretation of a palaeoclimate archive off NW Africa. \textit{Quaternary International, 196(1–2), 44–56}. https://doi.org/10.1016/j.quaint.2008.03.014
		
		\bibitem{Atlantic sediment cores last 40k years}
		Waelbroeck, C., Lougheed, B.C., Vazquez Riveiros, N. et al. Consistently dated Atlantic sediment cores over the last 40 thousand years. \textit{Sci Data 6, 165 (2019)}. https://doi.org/10.1038/s41597-019-0173-8
	\end{thebibliography}
	
\end{document}
