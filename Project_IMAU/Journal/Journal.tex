\documentclass[twocolumn,a4paper,aps,amsmath,amssymb,floatfix,superscriptaddress]{revtex4-2}
\usepackage{amsmath}
\usepackage{gensymb}
\usepackage{amsfonts}
\usepackage[a4paper, total={6in, 8in}]{geometry}
\usepackage{braket}
\usepackage{calligra}
\usepackage{cancel}
\usepackage{hhline}
\usepackage{tikz-feynman}
\usepackage{physics}
\usepackage{amsfonts}
\usepackage{makecell}
\usepackage{threeparttable}
\usepackage{multirow}
\usepackage{caption}
\usepackage{float}
\usepackage[caption = false]{subfig}
\usepackage{graphicx}     % Include figure files
\usepackage{dcolumn}		% Align table columns on decimal point
\usepackage{hyperref}     % To allow for hyperlinks in the document and to e.g. web sources
\usepackage{bm}
\usepackage[utf8]{inputenc}  % Allows use of één instead of \'e\'en.

\newcommand{\pref}{\frac{1}{4\pi\epsilon_0}}
\newcommand\at[3]{\left.#1\right|_{#2}^{#3}}
\renewcommand{\thesubsection}{\thesection.\roman{subsection}}
\renewcommand*{\thesubsubsection}{\thesubsection.\arabic{subsubsection}}
\renewcommand{\d}{\textnormal{d}}
\renewcommand{\r}{\textnormal{\large\textcalligra{r}}}
\newcommand{\del}{\mathbf{\nabla}}
\newcommand{\Dv}[3][1]{\frac{\textnormal{D}^{#1}#2}{\textnormal{D}#3^{#1}}}
\renewcommand{\vec}{\mathbf}

\begin{document}	
	\title{Lab Journal: analysis of an Atlantic clay sediment sample near the Canary Islands}
	\author{E. de Bie}
	\affiliation{Institute for Marine and Atmospheric Research Utrecht, \\
		Department of Physics, \\
		Faculty of Science, \\
		Utrecht University}
	
	\author{H. Hildebrand}
	\affiliation{Institute for Marine and Atmospheric Research Utrecht, \\
		Department of Physics, \\
		Faculty of Science, \\
		Utrecht University}
	
	\author{J. Gerlagh}
	\affiliation{Institute for Marine and Atmospheric Research Utrecht, \\
		Department of Physics, \\
		Faculty of Science, \\
		Utrecht University}
	\maketitle
	
	\section{Introduction}
	First, we shall discuss the points we are to research before starting our research plan. 
	In literature, we shall attempt to find:
	\begin{enumerate}
		\item the precipitation rate for this portion of the Atlantic, to determine the time resolution in our sample;
		\item perturbation of the ocean floor through natural or human mechanisms;
		\item concentration of microplastics, as an indication of the increase in nanoplastic concentration;
		\item amounts of detritus, and whether this would contaminate the measurement;
		\item protocols for extraction of the plastics from the clay.
	\end{enumerate}
	From this, we shall construct a protocol for our measurements and from there, we shall find the tools we may need to acquire.\\
	In general we split these into three categories: items 1 and 2 (to be researched by Joost) describe the context of the samples, items 3 and 4 (to be researched by Has) describe the contaminants, and item 5 (to be researched by Eva) describes the practical steps to be taken.
	
	\subsection{Question Answers}
	\subsubsection{}
	
	\subsubsection{}
	
	\subsubsection{Microplastic concentration as proxy for nanoplastic concentration}
	It was found that microplastic concentration is not a valid approximation for nanoplastic concentration. Microplastics arrive at the sea floor with different mechanisms and therefore do not have an exact relation to nanoplastics. \cite{Vertical flux of microplastic} \cite{MP in Atlantic deep waters} Microplastics behave like particles and are capable of sinking. Nanoplastics behave like colloids and do not sink quickly, unless they (self)aggregate. \cite{Aggregation of NP, what we know} The aggregates are too big to pass through the less porous filters. This can lead to significant loss of nanoplastics in the sample. The aggregates can be destroyed using sonication. \cite{Heteroaggregation, disaggregation} 
	\subsubsection{Significance of detritus in the sample}
	Detritus is present on the seafloor, as it is everywhere in the ocean. Detritus is also one of the things that the nanoplastics can aggregate on. As the aggregates are dispersed using sonication and afterwards the supernatant is filtered only very low mass organic compounds should remain in solution. These typically have a low boiling point so should not be detected at the same time as the nanoplastics (high boiling point). \cite{McMurry} If they do somehow come along with the nanoplastics (if the nanoplastics aggregate to them again during the rise in temperature perhaps) then their MS footprints should be sufficiently different as to not warrant any concern.
	\subsubsection{Existing Protocols}
	There exists research on the filtration, dialysis, and ultrafiltration retention rates using polystyrene nanospheres between 1000 and 50 nm.\cite{Assessment of nanoplastic extraction}
	
	\begin{thebibliography}{10}
		
		\bibitem{Assessment of nanoplastic extraction}
		Albignac, M., Maria, E., De Oliveira, T., Roux, C., Goudouneche, D., Mingotaud, A. F., Bordeau, G., \& ter Halle, A. (2023). \textit{Assessment of nanoplastic extraction from natural samples for quantification purposes.} Environmental Nanotechnology, Monitoring \& Management, 20. https://doi.org/10.1016/j.enmm.2023.100862
		
		\bibitem{Heteroaggregation, disaggregation}
		Lihua Li, Dan Luo, Shijie Luo, Jiale Yue, Xinzhi Li, Lianrong Chen, Xin Chen, Bowen Wen, Xitian Luo, Yongtao Li, Weilin Huang, Chengyu Chen,
		Heteroaggregation, disaggregation, and migration of nanoplastics with nanosized activated carbon in aquatic environments: Effects of particle property, water chemistry, and hydrodynamic condition,
		Water Research,
		Volume 266,
		2024,
		122399,
		ISSN 0043-1354,
		https://doi.org/10.1016/j.watres.2024.122399.
		(https://www.sciencedirect.com/science/article/pii/S0043135424012983)
		
		\bibitem{Vertical flux of microplastic}
		Emily Rowlands, Tamara Galloway, Matthew Cole, Victoria L. Peck, Anna Posacka, Sally Thorpe, Clara Manno,
		Vertical flux of microplastic, a case study in the Southern Ocean, South Georgia,
		Marine Pollution Bulletin,
		Volume 193,
		2023,
		115117,
		ISSN 0025-326X,
		https://doi.org/10.1016/j.marpolbul.2023.115117.
		(https://www.sciencedirect.com/science/article/pii/S0025326X23005490)
		Abstract: Estimated plastic debris floating at the ocean surface varies depending on modelling approaches, with some suggesting unaccounted sinks for marine plastic debris due to mismatches between plastic predicted to enter the ocean and that accounted for at the surface. A major knowledge gap relates to the vertical sinking of oceanic plastic. We used an array of floating sediment traps combined with optical microscopy and Raman spectroscopy to measure the microplastic flux between 50 and 150 m water depth over 24 h within a natural harbour of the sub-Antarctic island of South Georgia. This region is influenced by fishing, tourism, and research activity. We found a 69 \% decrease in microplastic flux from 50 m (306 pieces/m2/day) to 150 m (94pieces/m2/day). Our study confirms the occurrence of a vertical flux of microplastic in the upper water column of the Southern Ocean, which may influence zooplankton microplastic consumption and the carbon cycle.
		Keywords: Vertical transport; Fibres; Fragments; Floating sediment trap; Raman spectroscopy
		
		\bibitem{MP in Atlantic deep waters}
		Alicia Mateos-Cárdenas, Andrew J. Wheeler, Aaron Lim,
		Microplastics and cellulosic microparticles in North Atlantic deep waters and in the cold-water coral Lophelia pertusa,
		Marine Pollution Bulletin,
		Volume 206,
		2024,
		116741,
		ISSN 0025-326X,
		https://doi.org/10.1016/j.marpolbul.2024.116741.
		(https://www.sciencedirect.com/science/article/pii/S0025326X24007185)
		Abstract: This study explores microplastic and cellulosic microparticle occurrences in the NE Atlantic, focusing on the Porcupine Bank Canyon and Porcupine Seabight. Water samples from depths ranging between 605 and 2126 m and Lophelia pertusa coral samples from 950 m depth were analysed. Microparticles were detected in deep-water habitats, with concentrations varying from 2.33 to 9.67 particles L−1 in the Porcupine Bank Canyon, notably lower at greater depths. This challenges the assumption of deeper habitats solely acting as microplastic sinks. We also found evidence of microparticle adsorption and ingestion by L. pertusa. The presence of microparticles in cold-water corals underscores their vulnerability to pollutants. Furthermore, the dominance of rayon microparticles in both water and coral samples raises questions about marine pollution sources, potentially linked to terrestrial origins. This research emphasises the critical need for comprehensive exploration and conservation efforts in deep-sea environments, especially to protect vital ecosystems like L. pertusa reefs.
		Keywords: Microplastics; Deep-sea ecosystems; Anthropogenic pollutants; Cold-water corals; NE Atlantic
		
		\bibitem{Aggregation of NP, what we know}
		Alice Pradel, Charlotte Catrouillet, Julien Gigault,
		The environmental fate of nanoplastics: What we know and what we need to know about aggregation,
		NanoImpact,
		Volume 29,
		2023,
		100453,
		ISSN 2452-0748,
		https://doi.org/10.1016/j.impact.2023.100453.
		(https://www.sciencedirect.com/science/article/pii/S2452074823000046)
		Abstract: The presence of nanoplastics in the environment has been proven. There is now an urgent need to determine how nanoplastics behave in the environment and to assess the risks they may pose. Here, we examine nanoplastic homo- and heteroaggregation, with a focus on environmentally relevant nanoplastic particle models. We made a systematic analysis of experimental studies, and ranked the environmental relevance of 377 different solution chemistries, and 163 different nanoplastic particle models. Since polymer latex spheres are not environmentally relevant (due to their monodisperse size, spherical shape, and smooth surface), their aggregation behavior in natural conditions is not transferable to nanoplastics. A few recent studies suggest that nanoplastic particle models that more closely mimic incidentally produced nanoplastics follow different homoaggregation pathways than latex sphere particle models. However, heteroaggregation of environmentally relevant nanoplastic particle models has seldom been studied. Despite this knowledge gap, the current evidence suggests that nanoplastics may be more sensitive to heteroaggregation than previously expected. We therefore provide an updated hypothesis about the likely environmental fate of nanoplastics. Our review demonstrates that it is essential to use environmentally relevant nanoplastic particle models, such as those produced with top-down methods, to avoid biased interpretations of the fate and impact of nanoplastics. Finally, it will be necessary to determine how the heteroaggregation kinetics of nanoplastics impact their settling rate to truly understand nanoplastics' fate and effect in the environment.
		Keywords: Plastic pollution; Aggregation; Environment; Organic matter; Models; Experimentation; Microplastic
		
		\bibitem{McMurry}
		J.E. McMurray, Organic Chemistry, 9th Ed., p. 79.
		
		
		
	\end{thebibliography}
	
\end{document}
