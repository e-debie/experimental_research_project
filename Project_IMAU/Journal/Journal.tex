\documentclass[twocolumn,a4paper,aps,amsmath,amssymb,floatfix,superscriptaddress]{revtex4-2}
\usepackage{amsmath}
\usepackage{gensymb}
\usepackage{amsfonts}
\usepackage[a4paper, total={6in, 8in}]{geometry}
\usepackage{braket}
\usepackage{calligra}
\usepackage{cancel}
\usepackage{hhline}
\usepackage{tikz-feynman}
\usepackage{physics}
\usepackage{amsfonts}
\usepackage{makecell}
\usepackage{threeparttable}
\usepackage{multirow}
\usepackage{caption}
\usepackage{float}
\usepackage[caption = false]{subfig}
\usepackage{graphicx}     % Include figure files
\usepackage{dcolumn}		% Align table columns on decimal point
\usepackage{hyperref}     % To allow for hyperlinks in the document and to e.g. web sources
\usepackage{bm}
\usepackage[utf8]{inputenc}  % Allows use of één instead of \'e\'en.

\newcommand{\pref}{\frac{1}{4\pi\epsilon_0}}
\newcommand\at[3]{\left.#1\right|_{#2}^{#3}}
\renewcommand{\thesubsection}{\thesection.\roman{subsection}}
\renewcommand*{\thesubsubsection}{\thesubsection.\arabic{subsubsection}}
\renewcommand{\d}{\textnormal{d}}
\renewcommand{\r}{\textnormal{\large\textcalligra{r}}}
\newcommand{\del}{\mathbf{\nabla}}
\newcommand{\Dv}[3][1]{\frac{\textnormal{D}^{#1}#2}{\textnormal{D}#3^{#1}}}
\renewcommand{\vec}{\mathbf}

\begin{document}	
	\title{Lab Journal: analysis of an Atlantic clay sediment sample near the Canary Islands}
	\author{E. de Bie}
	\affiliation{Institute for Marine and Atmospheric Research Utrecht, \\
		Department of Physics, \\
		Faculty of Science, \\
		Utrecht University}
	
	\author{H. Hildebrand}
	\affiliation{Institute for Marine and Atmospheric Research Utrecht, \\
		Department of Physics, \\
		Faculty of Science, \\
		Utrecht University}
	
	\author{J. Gerlagh}
	\affiliation{Institute for Marine and Atmospheric Research Utrecht, \\
		Department of Physics, \\
		Faculty of Science, \\
		Utrecht University}
	\maketitle
	
	\section{Introduction}
	First, we shall discuss the points we are to research before starting our research plan. 
	In literature, we shall attempt to find:
	\begin{enumerate}
		\item the precipitation rate for this portion of the Atlantic, to determine the time resolution in our sample;
		\item perturbation of the ocean floor through natural or human mechanisms;
		\item concentration of microplastics, as an indication of the increase in nanoplastic concentration;
		\item amounts of detritus, and whether this would contaminate the measurement;
		\item protocols for extraction of the plastics from the clay.
	\end{enumerate}
	From this, we shall construct a protocol for our measurements and from there, we shall find the tools we may need to acquire.\\
	In general we split these into three categories: items 1 and 2 (to be researched by Joost) describe the context of the samples, items 3 and 4 (to be researched by Has) describe the contaminants, and item 5 (to be researched by Eva) describes the practical steps to be taken.
	
	\subsection{Question Answers}
	\subsubsection{}
	We found numerous different values for the sedimentation rates, influenced by exact location, period, etc. The found values were however all in the range of several mm \cite{Sediment distribution in the Atlantic} to cm \cite{Geomorphical investigaions NW Africa} per thousand years. So either way, assuming the pollution of the environment started, say, a century ago \cite{History of Plastic}, in theory only the upper millimeters of the ocean bottom should contain (traces of) plastic. Although that is unfortunately not the whole story.
	(There does perhaps seem to be more data, except we have not (yet?) figured out where exactly to find it or how to get it into a readable format, if we are even able to. Being this paper \cite{Atlantic sediment cores last 40k years}, but also the sites of several organisations, such as PLOCAN, BIOS and IODP)
	\subsubsection{}
	Something to take into consideration when analysing these samples, are possible perturbations caused by both human and natural causes. This could be for example anchors, strong water flows or (deep sea) animals mixing some of the upper layers. We found a source stating that the ocean floor in the area we're interested in is currently stable "Under static (gravitational) loading conditions" \cite{Sediment stability western Canary Islands}. However, human influences can also cause mixing and other perturbations in the sedimentation. \cite{Anthrophogenic influence sedimentation} \cite{Human impacts marine fossil record}. Unfortunately there seems to be no way to deal with this, other than maybe changing our hypothesis on the (maximum) depth we expect to find significant amounts of (nano)plastics.
	
	\subsubsection{Microplastic concentration as proxy for nanoplastic concentration}
	It was found that microplastic concentration is not a valid approximation for nanoplastic concentration. Microplastics arrive at the sea floor with different mechanisms and therefore do not have an exact relation to nanoplastics. \cite{Vertical flux of microplastic} \cite{MP in Atlantic deep waters} Microplastics behave like particles and are capable of sinking. Nanoplastics behave like colloids and do not sink quickly, unless they (self)aggregate. \cite{Aggregation of NP, what we know} The aggregates are too big to pass through the less porous filters. This can lead to significant loss of nanoplastics in the sample. The aggregates can be destroyed using sonication. \cite{Heteroaggregation, disaggregation} 
	\subsubsection{Significance of detritus in the sample}
	Detritus is present on the seafloor, as it is everywhere in the ocean. Detritus is also one of the things that the nanoplastics can aggregate on. As the aggregates are dispersed using sonication and afterwards the supernatant is filtered only very low mass organic compounds should remain in solution. These typically have a low boiling point so should not be detected at the same time as the nanoplastics (high boiling point). \cite{McMurry} If they do somehow come along with the nanoplastics (if the nanoplastics aggregate to them again during the rise in temperature perhaps) then their MS footprints should be sufficiently different as to not warrant any concern.
	\subsubsection{Existing Protocols}
	There exists research on the filtration, dialysis, and ultrafiltration retention rates using polystyrene nanospheres between 1000 and 50 nm.\cite{Assessment of nanoplastic extraction}\\
	From this research, we get several takeaways:\\
	Polymer membranes are not recommended, with retention rates far too high.\\
	"there was always retention even if the porosity was up to 50 times larger than the beads diameter. This implies that for pre-filtration, in the view to NP quantification in a real sample, the recoveries would be insufficient and very difficult to rationalize because of the high particle heterogeneity."\\
	Dialysis led to important losses of 50\% after 48h and 70\% after 72h.\\
	Ultrafiltration had losses of 35-40\%\\
	\\
	Next, I found a publication which determined metal fractionation of a marine sediment core from Antarctica.\cite{Geochemical characterization Antarctica}\\
	Again, there were several takeaways:\\
	Stainless steel was used to sample the core. It was subsampled using a Teflon blade. Only the inner part was analysed.\\
	To determine moisture content, a subsample was dried at 105° C until the mass was constant.\\
	The storage containers were cleaned extensively. They used a clean room (Class-100.000 / ISO 8, the "dirtiest" clean room\cite{cleanrooms}).\\
	\\
	Next, I analysed section 2 (`Materials and method') of a publication in which a cire sample was analysed from the Indian Ocea.\cite{Core Indian Ocean}
	In this, they used acetic acid to remove inorganic carbon contents of the sediment, and hydrogen peroxide to remove organic carbon contents. After this, they were washed with deionized water to remove remaining aces, peroxide, and salt.\\
	\\
	Finally, I read a paper of the PASADO core processing strategy, which appeared to have been made to allow several disciplines to perform research on the same core. Still, I extracted some points that I thought might be useful.\\
	Namely, the idea of leaving one half of the core (if split lengthways) for archiving purposes, if we have enough of the core to do that. Secondly, the idea to use a cutting tool with the right shape to not take the sides of the core, as those might be contaminated.\\
	\\
	In summary:\\
	We should try to avoid any filtering with porosity less than two orders of magnitude above 1 μm. If needed, stainless steel grids with cut off at 5 or 10 μm are preferred, with 15-20\% retention.\\
	We should use steel and glass tools wherever possible, as to not contaminate the sample.
	We clean the bottles using temperature. Verify if we can get a glovebox.\\
	Peroxide can be used to remove organic carbon contents, acetic acid can be used to remove inorganic carbon contents.\\
	A D-shaped cutter is useful as not to incorporate plastics from the storage medium. Furthermore, to use only one half of the sample may be a good idea for insurance's sake.
	\begin{thebibliography}{10}
		
		\bibitem{Assessment of nanoplastic extraction}
		Albignac, M., Maria, E., De Oliveira, T., Roux, C., Goudouneche, D., Mingotaud, A. F., Bordeau, G., \& ter Halle, A. (2023). \textit{Assessment of nanoplastic extraction from natural samples for quantification purposes.} Environmental Nanotechnology, Monitoring \& Management, 20. https://doi.org/10.1016/j.enmm.2023.100862
		
<<<<<<< HEAD\bibitem{Geochemical characterization Antarctica}
		Burgay, F., Abollino, O., Buoso, S., Costa, E., Giacomino, A., La Gioia, C., Garofalo, S. F., Pecoraro, G., \& Malandrino, M. (2020). \textit{Geochemical characterization of a marine sediment core from the Joides Basin, Ross Sea, Antarctica.} Marine Geology, 428. https://doi.org/10.1016/j.margeo.2020.106286 
				
		\bibitem{cleanrooms}
		American Cleanroom Systems. (2024, September 4). \textit{Clean room Classifications \& ISO Standards | American Cleanrooms Systems.} https://www.americancleanrooms.com/cleanroom-classifications/
		
		\bibitem{Heteroaggregation, disaggregation}
		Lihua Li, Dan Luo, Shijie Luo, Jiale Yue, Xinzhi Li, Lianrong Chen, Xin Chen, Bowen Wen, Xitian Luo, Yongtao Li, Weilin Huang, Chengyu Chen,
		Heteroaggregation, disaggregation, and migration of nanoplastics with nanosized activated carbon in aquatic environments: Effects of particle property, water chemistry, and hydrodynamic condition,
		Water Research,
		Volume 266,
		2024,
		122399,
		ISSN 0043-1354,
		https://doi.org/10.1016/j.watres.2024.122399.
		(https://www.sciencedirect.com/science/article/pii/S0043135424012983)
		
		\bibitem{Vertical flux of microplastic}
		Emily Rowlands, Tamara Galloway, Matthew Cole, Victoria L. Peck, Anna Posacka, Sally Thorpe, Clara Manno,
		Vertical flux of microplastic, a case study in the Southern Ocean, South Georgia,
		Marine Pollution Bulletin,
		Volume 193,
		2023,
		115117,
		ISSN 0025-326X,
		https://doi.org/10.1016/j.marpolbul.2023.115117.
		(https://www.sciencedirect.com/science/article/pii/S0025326X23005490)
		Abstract: Estimated plastic debris floating at the ocean surface varies depending on modelling approaches, with some suggesting unaccounted sinks for marine plastic debris due to mismatches between plastic predicted to enter the ocean and that accounted for at the surface. A major knowledge gap relates to the vertical sinking of oceanic plastic. We used an array of floating sediment traps combined with optical microscopy and Raman spectroscopy to measure the microplastic flux between 50 and 150 m water depth over 24 h within a natural harbour of the sub-Antarctic island of South Georgia. This region is influenced by fishing, tourism, and research activity. We found a 69 \% decrease in microplastic flux from 50 m (306 pieces/m2/day) to 150 m (94pieces/m2/day). Our study confirms the occurrence of a vertical flux of microplastic in the upper water column of the Southern Ocean, which may influence zooplankton microplastic consumption and the carbon cycle.
		Keywords: Vertical transport; Fibres; Fragments; Floating sediment trap; Raman spectroscopy
		
		\bibitem{MP in Atlantic deep waters}
		Alicia Mateos-Cárdenas, Andrew J. Wheeler, Aaron Lim,
		Microplastics and cellulosic microparticles in North Atlantic deep waters and in the cold-water coral Lophelia pertusa,
		Marine Pollution Bulletin,
		Volume 206,
		2024,
		116741,
		ISSN 0025-326X,
		https://doi.org/10.1016/j.marpolbul.2024.116741.
		(https://www.sciencedirect.com/science/article/pii/S0025326X24007185)
		Abstract: This study explores microplastic and cellulosic microparticle occurrences in the NE Atlantic, focusing on the Porcupine Bank Canyon and Porcupine Seabight. Water samples from depths ranging between 605 and 2126 m and Lophelia pertusa coral samples from 950 m depth were analysed. Microparticles were detected in deep-water habitats, with concentrations varying from 2.33 to 9.67 particles L−1 in the Porcupine Bank Canyon, notably lower at greater depths. This challenges the assumption of deeper habitats solely acting as microplastic sinks. We also found evidence of microparticle adsorption and ingestion by L. pertusa. The presence of microparticles in cold-water corals underscores their vulnerability to pollutants. Furthermore, the dominance of rayon microparticles in both water and coral samples raises questions about marine pollution sources, potentially linked to terrestrial origins. This research emphasises the critical need for comprehensive exploration and conservation efforts in deep-sea environments, especially to protect vital ecosystems like L. pertusa reefs.
		Keywords: Microplastics; Deep-sea ecosystems; Anthropogenic pollutants; Cold-water corals; NE Atlantic
		
		\bibitem{Aggregation of NP, what we know}
		Alice Pradel, Charlotte Catrouillet, Julien Gigault,
		The environmental fate of nanoplastics: What we know and what we need to know about aggregation,
		NanoImpact,
		Volume 29,
		2023,
		100453,
		ISSN 2452-0748,
		https://doi.org/10.1016/j.impact.2023.100453.
		(https://www.sciencedirect.com/science/article/pii/S2452074823000046)
		Abstract: The presence of nanoplastics in the environment has been proven. There is now an urgent need to determine how nanoplastics behave in the environment and to assess the risks they may pose. Here, we examine nanoplastic homo- and heteroaggregation, with a focus on environmentally relevant nanoplastic particle models. We made a systematic analysis of experimental studies, and ranked the environmental relevance of 377 different solution chemistries, and 163 different nanoplastic particle models. Since polymer latex spheres are not environmentally relevant (due to their monodisperse size, spherical shape, and smooth surface), their aggregation behavior in natural conditions is not transferable to nanoplastics. A few recent studies suggest that nanoplastic particle models that more closely mimic incidentally produced nanoplastics follow different homoaggregation pathways than latex sphere particle models. However, heteroaggregation of environmentally relevant nanoplastic particle models has seldom been studied. Despite this knowledge gap, the current evidence suggests that nanoplastics may be more sensitive to heteroaggregation than previously expected. We therefore provide an updated hypothesis about the likely environmental fate of nanoplastics. Our review demonstrates that it is essential to use environmentally relevant nanoplastic particle models, such as those produced with top-down methods, to avoid biased interpretations of the fate and impact of nanoplastics. Finally, it will be necessary to determine how the heteroaggregation kinetics of nanoplastics impact their settling rate to truly understand nanoplastics' fate and effect in the environment.
		Keywords: Plastic pollution; Aggregation; Environment; Organic matter; Models; Experimentation; Microplastic
		
		\bibitem{McMurry}
		J.E. McMurray, Organic Chemistry, 9th Ed., p. 79.
		
		\bibitem{Core Indian Ocean}
		Sensarma, S., Gupta, S.M., Banerjee, R. et al. \textit{Change of lithofacies in marine sediment core from Quaternary to Pre-Quaternary: A case study from the Central Indian Ocean} Basin. J Earth Syst Sci 129, 54 (2020). https://doi.org/10.1007/s12040-019-1318-z 
		
		\bibitem{PASADO}
		Ohlendorf, C., Gebhardt, C., Hahn, A., Kliem, P., \& Zolitschka, B. (2011). \textit{The PASADO core processing strategy — A proposed new protocol for sediment core treatment in multidisciplinary lake drilling projects.} Sedimentary Geology, 239(1-2), 104–115. https://doi.org/10.1016/j.sedgeo.2011.06.007
		
		\bibitem{Anthrophogenic influence sedimentation}
		Cheng, Z., Wang, X. H., Jalón-Rojas, I., \& Liu, Y. (2019). Reconstruction of sedimentation changes under anthropogenic influence in a medium-scale estuary based on a decadal chronological framework. Estuarine Coastal and Shelf Science, 227, 106295. https://doi.org/10.1016/j.ecss.2019.106295
		
		\bibitem{Sediment distribution in the Atlantic} Ewing, M., Carpenter, G., Windisch, C., \& Ewing, J. (1973). Sediment distribution in the oceans: the Atlantic. Geological Society of America Bulletin, 84(1), 71. https://doi.org/10.1130/0016-7606(1973)84
		
		\bibitem{Human impacts marine fossil record}
		Nawrot, R., Zuschin, M., Tomašových, A., Kowalewski, M., \& Scarponi, D. (2024). Ideas and perspectives: Human impacts alter the marine fossil record. Biogeosciences, 21(9), 2177–2188. https://doi.org/10.5194/bg-21-2177-2024
		
		\bibitem{Sediment stability western Canary Islands} Roberts, J. A., \& Cramp, A. (1996). Sediment stability on the western flanks of the Canary Islands. Marine Geology, 134(1–2), 13–30. https://doi.org/10.1016/0025-3227(96)00021-7
		
		\bibitem{History of Plastic} Verney, F. (2023, November 13). The history of plastic in 15 key dates. Carbiolice. https://www.carbiolice.com/en/news/the-history-of-plastic-in-15-key-dates-2/
		
		\bibitem{Geomorphical investigaions NW Africa} Von Suchodoletz, H., Faust, D., \& Zöller, L. (2008). Geomorphological investigations of sediment traps on Lanzarote (Canary Islands) as a key for the interpretation of a palaeoclimate archive off NW Africa. Quaternary International, 196(1–2), 44–56. https://doi.org/10.1016/j.quaint.2008.03.014
		
		\bibitem{Atlantic sediment cores last 40k years}
		Waelbroeck, C., Lougheed, B.C., Vazquez Riveiros, N. et al. Consistently dated Atlantic sediment cores over the last 40 thousand years. Sci Data 6, 165 (2019). https://doi.org/10.1038/s41597-019-0173-8
	\end{thebibliography}
	
\end{document}
