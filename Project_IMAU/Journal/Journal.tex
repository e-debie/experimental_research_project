\documentclass[twocolumn,a4paper,aps,amsmath,amssymb,floatfix,superscriptaddress]{revtex4-2}
\usepackage{amsmath}
\usepackage{gensymb}
\usepackage{amsfonts}
\usepackage[a4paper, total={6in, 8in}]{geometry}
\usepackage{braket}
\usepackage{calligra}
\usepackage{cancel}
\usepackage{hhline}
\usepackage{tikz-feynman}
\usepackage{physics}
\usepackage{amsfonts}
\usepackage{makecell}
\usepackage{threeparttable}
\usepackage{multirow}
\usepackage{caption}
\usepackage{float}
\usepackage[caption = false]{subfig}
\usepackage{graphicx}     % Include figure files
\usepackage{dcolumn}		% Align table columns on decimal point
\usepackage{hyperref}     % To allow for hyperlinks in the document and to e.g. web sources
\usepackage{bm}
\usepackage[utf8]{inputenc}  % Allows use of één instead of \'e\'en.

\newcommand{\pref}{\frac{1}{4\pi\epsilon_0}}
\newcommand\at[3]{\left.#1\right|_{#2}^{#3}}
\renewcommand*{\thesubsubsection}{\thesubsection.\alph{subsubsection}}
\renewcommand{\d}{\textnormal{d}}
\renewcommand{\r}{\textnormal{\large\textcalligra{r}}}
\newcommand{\del}{\mathbf{\nabla}}
\newcommand{\Dv}[3][1]{\frac{\textnormal{D}^{#1}#2}{\textnormal{D}#3^{#1}}}
\renewcommand{\vec}{\mathbf}

\begin{document}	
	\title{Lab Journal: analysis of an Atlantic clay sediment sample near the Canary Islands}
	\author{E. de Bie}
	\affiliation{Department of Physics, \\
		Utrecht University}
	
	\author{H. Hildebrand}
	\affiliation{Department of Physics, \\
		Utrecht University}
	
	\author{J. Gerlagh}
	\affiliation{Department of Physics, \\
		Utrecht University}
	\maketitle
	
	First, we shall discuss the points we are to research before starting our research plan. 
	In literature, we shall attempt to find:
	\begin{itemize}
		\item the precipitation rate for this portion of the Atlantic, to determine the time resolution in our sample;
		\item Concentration of microplastics, as an indication of the increase;
		\item Perurbation of the ocean floor through natural or human mechanisms;
		\item Protocols for extraction of the plastics from the clay;
		\item Amounts of detritus, and whether this would contaminate the measurement
	\end{itemize}
	From this, we shall construct a protocol for our measurements and from there, we shall find the tools we may need to acquire.
\end{document}
